%% Resumo

\chapter{Resumo}
\label{ch:resumo}

Neste trabalho foi elaboramos um modelo para a construção de estruturas sociais em relação a características cognitivas das pessoas.
Com base em estudos nos campos da neurociência e da psicologia social, além do uso das técnicas de mecânica estatística e da teoria de aprendizado de máquina, fomos capazes de estabelecer um modelo de agentes que trocam opiniões, aprendem e escolhem seus parceiros com base nas informações que obtêm.

A dinâmica das relações sociais introduzida aqui é capaz de produzir complexidade na estrutura das relações sociais e na distribuição de opiniões de forma correlacionada.
Essa dinâmica nos dá um parâmetro que mede o nível de organização social e nos permite interpretar o fenômeno de formação de comunidades como uma estrutura que requer uma \emph{"massa crítica"} para se estabelecer.

Com base nesses resultados, propusemos um modelo capaz de reproduzir parte do comportamento observado no plenário brasileiro ao londo do anos em relação a votações de projetos de lei sob diferentes mandatos presidenciais.

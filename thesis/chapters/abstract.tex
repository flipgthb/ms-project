%% Abstract

\chapter{Abstract}
\label{ch:abstract}

In this work we establish a model for social structure construction related with human cognitive character.
From neuroscience and social psychology studies, and also relying on technique from statistical mechanics and machine leaning theory, we were able to establish a model of agents who exchange opinions, learn together and choose their partnerships using the information they gather.

The social relationship dynamics here introduced is capable of generating complexity on the social structure level and a on opinion distribution in a correlated way.
This dynamics gives a parameter representing the degree to which society is organized and allow us to see the phenomena of community formation as a structure requiring a \emph{"critical mass"} to fixate.

These results guided us in proposing a model able to replicate a portion of the behavior observed at the Brazilian plenary through year of voting law projects under different president mandates.

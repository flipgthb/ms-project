\documentclass[12pt]{article}
\usepackage[utf8]{inputenc}
\usepackage[T1]{fontenc}
\usepackage[brazil, english]{babel}
\usepackage{graphicx}
\usepackage{setspace}
\usepackage{geometry}
\usepackage{indentfirst}
\usepackage[small,bf]{caption}


\begin{document}
\selectlanguage{brazil}
\begin{titlepage}
\begin{center}
\textsc{\Large projeto de pesquisa - mestrado\\}
%\textsc{\small Fundação de Amparo à Pesquisa do Estado de São Paulo - FAPESP}\\[6.5cm]
\textsc{\Huge Quebra de simetria espontânea, Limites Cognitivos e  Complexidade de Sociedades}\\[0.5cm]
\textsc{\small {\it Estudante:} felippe alves pererira\\ {\it Orientador:} nestor caticha}
\vfill
\textsc{universidade de são paulo - instituto de física}\\
\textsc{\small departamento de física geral}\\
\small \today
\end{center}
\end{titlepage}

\tableofcontents
\thispagestyle{empty}

\begin{abstract}
O estudo de fenômenos coletivos e transições de fase tem sido central,
desde o começo do século passado, em áreas tradicionais da física. 
Nas últimas  décadas tem ficado claro, a partir do trabalho de Jaynes e outros
autores [1-6] que o formalismo da Mecânica Estatística
pode ser descrito dentro de um contexto de teoria de informação. A partir desta
constatação o horizonte de aplicabilidade de Mecânica Estatística foi 
enormemente aumentado, passando a incorporar a possibilidade de descrever 
sistemas de processamento de informação. Nesta categoria podemos citar como exemplos
modelos inspirados em sistemas sensoriais periféricos (visual, auditivo, olfativo, etc.), 
modelos de memórias associativas, aprendizagem de máquinas, processamento de 
informação distribuidos em sistemas 
de agentes, entre outros.

Este projeto está inserido num esforço interdisciplinar onde aplicamos 
idéias e técnicas de teoria de informação para descrever modelos de 
estruturas sociais onde há fenômenos coletivos e transições de fase.
Sociedades de agentes que aprendem e trocam informação são construidas usando
dados de neurociência e teoria de aprendizagem de máquinas e são analisados usando 
técnicas analíticas e numéricas de Mecânica Estatística e de inferência probabilística.  
Como exemplo de tais transições
temos trabalhado com estudos da quebra da organização egalitária de sociedades 
que é considerado  um problema central  em diversas áreas de estudo, como antropologia 
e economia. Outro exemplo diz respeito ao problema de construção de estados onde a 
sociedade apresenta ordem ou desordem de crenças sobre questões discutidas no âmbito
da sociedade. Estilos cognitivos diferentes dos agentes levam a sociedades
com diferentes assinaturas estatísticas e estas são comparadas com assinaturas extraídas de
questionários respondidos por milhares de pessoas.  

Neste trabalho 
desenvolveremos e estudaremos  modelos de sociedade onde há fontes conflitante de informações. 
Estudaremos o efeito da  diferença da compĺexidade computacional  entre as fontes de informação 
e dos agentes. 

Além da metodologia de teoria de informação o estudante deverá aprender teoria de análise de 
instituições em ciências sociais. A metodologia mais relevante nesta área foi introduzida por
Elinor Ostrom e colaboradores \cite{Ostrom} para desconstruir nos elementos básicos
 instituições
que recebem o nome de {\it commons} e para entender como pode ser evitada 
em casos específicos a tragédia dos comuns\cite{Hardin}.


\end{abstract}

\selectlanguage{english}
\begin{abstract}
The study of collective and critical phenomena has been central in many
areas of Physics since the begining of the last century. In the last few decades,
it has been shown by Jaynes and other authors [1-6], that the formalism of Statistical 
Mechanics can be described within the context of Information theory. This
enabled the scope of applications of Statistical Mechanics to be wildly expanded,
and now it is used to address the study of information processing systems. 
In this area we can mention among others, such  examples  as peripheral
sensory systems (visual, auditive, olfactive, etc), associative memories, 
machine learning and information processing agent systems.

This project is imersed in a wider scope interdisciplinary effort, where
we use ideas and techniques from information theory to describe models of social structures
where there are collective  phenomena and phase transitions. 
Societies of agents that learn and exchange information are built using
neuroscience data, machine learning theory and are analyzed using analytical
and numerical techniques drawn from Statistical Mechanics and probabilistic inference.
As examples of such transitions we have worked with a central problem in social sciences
antropology and economics, that deals with the transitions from hierarchical 
to egalitarian societies. In another example  we dealt with the construction 
of states where the society presents order or disorder when measured by the 
distribution of opinions about certain issues. Different  cognitive styles lead
to societies with different statistical signatures and these were compared to
the correspondent statistical signatures extracted from data obtain from 
thousands of questionnaries answered by real people. 

In these work we will study models of societies in the presence of conflicting 
infromation sources. We will study effects of the difference of computational 
complexity of the sources and of the agents.

In addition to infromation theory, the student will have to learn about
methodolgy for the  analysis of institutions in the social sciences. 
The most relevant methodolgy in this area was introduced by Elinor Ostrom
and collaborators  \cite{Ostrom}, for deconstructing the type of institutions
that are known as {\it commons} into their basic elements and to understand how to avoid,
from the analysis of specific cases how to avoid the tragedy of the commons \cite{Hardin}.

\end{abstract}

\selectlanguage{brazil}
\setcounter{page}{3}
\doublespace
\section{Introdução}
\subsection{Modelos de sociedades, crenças compartilhadas
 e Teoria de Informação}
Um problema central no estudo de sociedades é a questão de formação
de padrões de crenças coletivas. Este tipo de fenômeno está por trás da
formação de capital social e tradições culturais.
Estamos interessados na modelagem de sociedades humanas
usando técnicas de Mecânica Estatística e Teoria de Informação, com dados  
obtidos por estudos nas áreas de arqueologia, antropologia cultural 
e ecológica, economia e neurociência.  



Recentemente foi introduzido \cite{CaVi} um modelo de um sistema de 
agentes que trocam informação baseado em evidências psicofísicas 
e neurocognitivas sobre a possibilidade de existência de diferentes
estilos cognitivos. Os resultados mostram que as assinaturas estatísticas
de agentes com diferentes estilos cognitivos são semelhantes às
assinaturas empíricas de diferentes grupos de afiliação política, 
obtidas a partir de
extensos bancos de dados coletados pelo grupo de J. Haidt. 
Ao percorrer o espectro
de afiliação política da esquerda para a direita em relação a questões
sociais (e não econômicas) e sua base moral, há uma mudança
no estilo de lidar com a informação nova versus informação corroborativa.
Aqueles agentes que dão mais importância a informação nova que a
informação corroborativa, são estatisticamente semelhantes a, no contexto
dos EUA, liberais. Os que dão a mesma importância são semelhantes 
estatisticamente a conservadores. Além disso os tempos característicos para
atingir o equílibrio após a perturbações do sistema depende do estilo 
cognitivo. Grupos de agentes identificados como conservadores demoram mais
para voltar ao equilíbrio e os liberais menos. Isso apóia a conclusão 
que a base para os nomes das classificações políticas, inspirados em
atitudes e estilos, decorrem dos estilos neurocognitivos
subjacentes. 

Também realiamos o
 estudo de sistemas de agentes que usam algoritmos de aprendizagem {\it on-line}
otimos, no sentido que se aproximam o máximo possível de algortimos Bayesiano,
e são chamados de agentes Bayesianos. Em semelhança a seres humanos, postulamos
ao menos duas fases no desenvolvimento cognitivo. Na primeira fase
há um aprendizagem Bayesiano. O algoritmo muda com o número de eventos
sociais de trocas de informação até o fim da primeira fase, isso é
típico do algoritmo Bayesiano que se adapta: o agente aprende a aprender. Durante a
segunda fase o algoritmo está congelado. Embora o agente possa aprender, isto é 
há adaptabilidade do agente, seu estilo cognitivo permance congelado e 
reflete a sua história durante a primeira fase. O resultado novamente
 é que as assinaturas estatísticas refletem o estilo cognitivo. Agora
ha uma correlação entre o número de encontros sociais e a  assinatura estatística.
Por outro lado, comparando com dados sobre pessoas que declararam sua 
afiliação política, podemos ver uma correlação positiva entre número de encontros sociais
na primeira fase dos agentes e afiliação política liberal. 
Este resultado é de extremo interesse dado que em 
\cite{Settle2010a}  encontraram o mesmo tipo de correlação em dados sobre seres humanos
para portadores de duplo alelo do gene DRD4-7r. Esse trabalho 
mostra pela primeira vez uma interação direta entre ideologia política
e genética, e nós mostramos as bases da teoria de informação para esse mecanismo.


\newpage
\section{Projeto e Objetivos}
Uma descrição física do problema dinâmico de agentes interagentes pode ser feita  
como o  resultado de aprendizagem por uma dinâmica ao
longo de um gradiente de uma função de Lyapunov mais uma componente estocástica.
Em geral a função de Lyapunov tem termos que se originam de duas fontes principais.
A estrutura desta função depende  de neurobiologia e teoria de aprendizado 
e dos tipos de interações sociais do sistema a ser modelado. A partir
de idealizações deste tipo de consideração obtemos uma dinâmica particular.

Neste projeto iremos considerar as interações entre os agentes mas
também adicionaremos um termo de interação com o meio ambiente. 
Assim será possível avaliar o efeito de 
\begin{enumerate}
\item Variação da complexidade do meio ambiente
\item Dependência na estrutura social frente a ameaças externas
\item Variação da importância para a sobrevivência do domínio do meio ambiente complexo.
\item mudanças de parâmetros da teoria, estudo de transições de fase.
\end{enumerate}

Há vários cenários empíricos que podem ser modelados com este tipo de estrutura metodológica, entre eles mencionamos:

\begin{enumerate}
\item Surgimento de facções com opiniões diferentes
\item Modelos para o surgimento de proto-religiões
\item Evolução cultural
\end{enumerate}


Uma parte ambiciosa do projeto é considerar a possiblidade de
usar o esquema de análise de Ostrom \cite{Ostrom} para descrição dos 
{\it commons}, pelo qual recebeu o Prêmio Nobel de Economia. Este sistema
de análise foi desenvolvido para sistemas de interação humana em meios 
ambientes biofísicos. O exemplo paradigmático é sobre exemplos de emergência
de regras de interação que levam a evitar a catástrofe dos comuns, por
exemplo a pesca de lagostas em Maine, EUA. 
Estamos interessados porém na emergência de tais estruturas, não num ambiente
biofísico mas cultural. A extensão da teoria para ambientes informacionais
foi feita por  Madison, Frischmann e 
Strandburg \cite{Strandburg}. Isto está bem além  das necessidade para um mestrado,
porém é natural que o pesquisador se defronte com problemas acima 
da sua capacidade e desenvolver a atitude correta frente a 
desafios faz parte da formação que deve ser adquirida já no mestrado.
\subsection{Objetivos}
O objetivo do projeto é que o estudante participe na elaboração detalhada
do modelo de agentes e no seu estudo numérico e analítico. Também deverá
participar na comparação dos resultados com dados empíricos.

O desenvolvimento deste projeto envolverá também o Prof. Renato Vicente
do IME-USP, com quem temos colaborado neste e outros tópicos no passado.

\newpage
\section{Cronograma de Atividades}
Durante o desenvolvimento do  projeto de mestrado o estudante deverá
estudar as técnicas de Mecânica Estatística, Teoria de Informação
e Teoria de Aprendizagem de Máquinas, para poder contribuir
na construção de variações sobre o modelo, escrever programas de simulação
e realizar cálculos analíticos das quantidades de interesse.

Há também a necessidade de vasta leitura de literatura na área de antropologia, economia, arqueologia e primatologia. Estas leituras serão realizadas predominantemente durante os meses de janeiro a julho de 2012. Exemplos destas leituras aparecem na bibliografia.

\subsection{Primeiro Semestre - 2012}
\begin{itemize}
\item Disciplinas de Pós-Graduação - Mecânica Quântica I (IFUSP),
Fundamentos Matemáticos da Relatividade Geral (IFUSP), Simulação de processos Físicos com Método de Monte Carlo (IFUSP)
\item Estudos da bibliografia relevante ao problema
\item Participação da elaboração dos programas de simulação
\end{itemize}

\subsection{Segundo Semestre - 2012}
\begin{itemize}
\item Desenvolvimento e análise dos modelos de agentes
\item Elaboração dos programas para a simulação dos modelos
\item Uma disciplina de pós-graduação a ser escolhida
\end{itemize}


\subsection{Primeiro Semestre - 2013}
\begin{itemize}
\item Obtenção de resultados e análise teórica
\item Análise estatística dos resultados das simulações e tratamento analítico.
Comparação com dados etnográficos e econômicos
\end{itemize}


\subsection{Segundo Semestre - 2013}
\begin{itemize}
\item Elaboração da dissertação
\item Defesa
\end{itemize}



\newpage
\begin{thebibliography}{30}

\bibitem{Jaynes}E. T. Jaynes {\it Probability theory: The Logic of
    Science} Cambridge University Press, Cambridge (2003)
\bibitem{Ariel} A. Caticha Lecture Notes on information Theory,
\bibitem{mackay} MacKay D.J.C., \textit{Information Theory, Inference
    and Learning Algorithms}
Cambridge University Press, Cambridge, (2003).
\bibitem{Parisi} G. Parisi {\it et al, Spin Glasses and Beyond}
\bibitem{Engel} A. Engel and C. van den Broeck {\it Statistical
    Mechanics of Learning}
\bibitem{Nishimori} Nishimori, Statistical Physics of Spin Glasses and Information Processing: An Introduction

\bibitem{Ostrom}Ostrom, Elinor (1990). Governing the Commons: The Evolution of Institutions for Collective Action. Cambridge University Press. ISBN 0-521-40599-8; Ostrom, Elinor (July 2009). "A General Framework for Analyzing Sustainability of Social-Ecological Systems". Science 325: 419–422.
\bibitem{Hardin}Garrett Hardin
"The Tragedy of the Commons". Science 162 (3859): 1243–1248. 1968.

\bibitem{Strandburg}Michael J. Madison, Brett M. Frischmann
and Katherine J. Strandburg, 
Legal Studies Research Paper Series
Working Paper No. 2008-26
August 2008

\bibitem{Estes}William K. Estes, Research and Theory on the
Learning of Probabilities,
Journal of the American Statistical Association, Vol. 67, No. 337
(Mar., 1972), pp. 81-102

\bibitem{CaVi}N.Caticha and R.Vicente ¨Agent-based social psycology: from neurocognitive processes to social data¨ Volume: 14, Issue: 5(2011) pp.711-731 Advances in Complex Systems (ACS)

\bibitem{Settle2010a}Settle, Jaime E Dawes, Christofer T Christakis, Nicholas A Fowler, James H (2010) 72,4 1189 ¨Friendships Moderates an Association Between a Dopamine Gene and Political Ideology.¨ The journal of politicsz

\bibitem{Kandel} Kandel, et al, Essentials of neural sciences and behavior.


\bibitem{Dunbar} R. Dunbar,  Grooming, Gossip, and the
Evolution of Language. Faber, London (1997)



\bibitem{Eisenberger} Eisenberger, N., Lieberman, M., and Williams, K., Does rejection hurt? An fMRI
study of social exclusion, Science 302 (2003) 290-292.

\bibitem{Mazur} Allan Mazur,
Biosociology
of Dominance
and Deference
ROWMAN and LITTLEFIELD PUBLISHERS, INC.
(2005

 \bibitem{Pumain}
Hierarchy in Natural
and Social Sciences
edited by
Denise Pumain, Springer (2006)

\bibitem{Bohem}Christopher Boehm
Hierarchy in the Forest
The Evolution of
Egalitarian Behavior, Harvard University Press (2001)

\bibitem{Earle} Timothy Earle,
How Chiefs came to power, Stanford University Press (1997)

\bibitem{Salzman} Philip Carl Salzman, Pastoralists: Equality, Hierarchy and
the State. Westview Press (2004)


\bibitem{Fleagle}Edited by
John G. Fleagle
Charles H. Janson
Kaye E. Reed, Primate Communities
Cambridge University Press (2004)

\bibitem{Wason} Paul K. Wason, The archaeology of rank
Cambridge University Press (1994)



\bibitem{Sassaman} Kenneth E. Sassaman, Journal of Archaeological Research Vol 12 september 2004

\bibitem{Menger} Carl Menger
On the Origins of Money
Economic Journal, volume 2,(1892) p. 239-55.
translated by C.A. Foley

\bibitem{Maisels}Charles Keith Maisels, The Emergence of
Civilization:
From hunting and gathering to
agriculture, cities, and the state in the
Near East,
Routledge   (2005)

\bibitem{CatichaetalA} N. Caticha, R. Calsaverini, R. Vicente, {\it
Cognitive limits and Breakdown of the Egalitarian Society} preprint   (2012)

\bibitem{Schonmannetal2011a} R. Schonmann, R. Vicente e N. Caticha
Two-level Fisher-Wright framework with selection and migration: An approach to studying evolution in group structured populations.
arXiv:1106.4783 (2011)

\bibitem{Perreault}C. Perreault, C. Moya, and R. Boyd. A Baysian approach to the evolution of social learning. Evolution and Human Behavior, Proofs posted online April 2012

\end{thebibliography}
\end{document}





A descrição da teoria abaixo é muito abreviada.

O estudo de aparecimento de altruísmo em modelos evolucionários
tem sido bastante discutido na literatura. Recentemente foi mostrado
\cite{Schonmannetal2011a} que é possível o aparecimento de genes
chamados altruístas em populações estruturadas em grupos sob
regimes de migração compatíveis com estimativas baseadas em dados empíricos.
Um ingrediente importante para isto é que haja estruturas de colaboração
e punição entre subgrupos que formam coligações e proto-instituições.
Um gene associado ao altruísmo, neste contexto, significa que seu portador
terá uma ajuste ({\it fitness}) menor, mas os membros do grupo serão beneficiados
e terão um ajuste maior.
Exemplos destas coligações são encontradas frequentemente em humanos mas
também abundam em chimpanzés, bonobos e outros primatas.

Os agentes destas sociedades apresentam uma tendência à conformidade e
à troca de favores. Estes aparecem na forma de catação ({\it grooming})
ou troca de comida, sexo e alianças. Em humanos estão associados, além
destes, à troca de informação na forma de fofocas ({\it gossip}) \cite{Dunbar}.
É interessante notar que o repagamento de favores recebidos não precisa
ser imediato. A probabilidade de repagamento decai lentamente com o tempo, com
tempos característicos que dependem da espécie. O repagamento também não precisa necessariamente ser na mesma
categoria de mercadoria ou favor. Por exemplo, de Waal mostra exemplos em chimpanzés onde
pode haver troca de sexo por defesa, ou de catação por carne.
Humanos extraem prazer de tal troca de favores, e ser deixado de lado leva à
ativação de regiões associadas à dor \cite{Eisenberger}.
A dor da injustiça na troca leva a um consenso na intercambialidade das
diferentes mercadorias.
Em \cite{Bohem} há uma descrição do ponto de vista primatológico, de como
estas proto-instituições levam a um convívio onde há de graus moderados de
hierarquia (chimpanzés e bonobos) a sociedades egalitárias (humanos).
Com o começo da agricultura e pastoralismo houve um aumento do tamanho dos
bandos humanos. Para isto contribuíram diversificação das fontes de alimento,
melhores métodos de produção e estocagem, menor necessidade
de migrações e uma diminuição do intervalo de idade entre irmãos. Segue que
há uma transição, sob condições cognitivas fixas, devido ao aumento do
tamanho do bando \cite{CatichaetalA} para uma sociedade estruturada.
A quebra de simetria na representação da estrutura social é o início de uma sociedade hierárquica.
As figuras \ref{individualgraph} e \ref{degree} mostram resultados de simulações
de Monte Carlo ilustrando a quebra de simetria.

\begin{figure}[h]
\center
%\subfigure{
\includegraphics[scale=0.3, angle=0]{fullgraph20.png}%}
%\subfigure{
\includegraphics[scale=0.3, angle=0]{powerstruggle20.png}%}
%\subfigure{
\includegraphics[scale=0.3, angle=0]{stargraph20.png}%}
\caption{Esquerda: Gráfo inteiro, recursos cognitivos são suficientes
para considerar todos os pares de interações.
 Centro: Grafo diluído, os recursos cognitivos são reduzidos e alguma informação
é perdida. Direita: Uma redução maior leva à quebra de simetria e um agente
é escolhido como o central.
 }\label{individualgraph}
\end{figure}


A introdução de linguagem (fofoca) leva ao aparecimento de correlações entre
as representações sociais dos diferentes agentes. A probabilidade do agente
central da hierarquia ser o mesmo para os diferentes agentes,
aumenta com a taxa de troca de informação.

Postulando que a probabilidade de troca de favores entre dois agentes
decai com a distância no grafo de representações sociais, temos
que há um agente, o central, que participa com muito maior
probabilidade de trocas com membros do grupos do que qualquer outro agente.

A teoria de Herrman-Pilliath introduz o conceito de droga perceptual. Ele
considera uma atividade que originalmente pode ter valor evolutivo. Agentes que
tem prazer, por ativação de circuitos dopaminérgicos, realizam essa tarefa
com grande probabilidade. A busca hedônica, de pura ativação dopaminérgica, sem
o necessário ganho evolutivo original, leva ao aumento da frequência de realização
da atividade. As trocas de favores, necessárias para amálgama das
coligações, podem passar, por esta espécie de corrida armamentista, a ser
realizadas de forma mais frequente. Numa sociedade hierárquica, o membro central
realizará mais trocas e dificilmente poderá repagar os favores imediatamente.
Deverá haver alguma forma de crédito que leva à demora de pagamento e
a quebra de promessas de repagamento levará a atritos sociais a não ser que
haja um meio de aumentar a memória de pagamento de tais dívidas.
Surge assim uma mercadoria, que não necessita ter valor intrínseco,
como meio de manter registro de trocas. Surge assim o dinheiro. A existência
deste meio permite a diversificação da sociedade e aumento da produtividade
através de mecanismos já estabelecidos.
 
\begin{figure}[h]
\center
\hspace{1cm}\includegraphics[scale=0.4]{emergencesStarGraph4.png}
\caption{Grau médio do grafo de representação social como função
$N(N-1)/(2\alpha)$, que  mede os
 recursos cognitivos $\alpha$ (aumenta para a esquerda) e do número
 de membros do
 grupo (aumenta para a direita). Em azul aparece o grau máximo e em preto o grau médio.}
\label{degree}
\end{figure}



s e alianças. Devido à competição entre
as limitações cognitivas associadas à memória finita dos agentes e às vantagens associadas ao conhecimento
das proto-instituições sociais existentes, se observa uma transição de regime de organização
social quando o número de componentes do bando ultrapassa um limite.

Este limite pode estar associado ao {\it número de Dunbar}, que descreve
uma lei de potência para a relação entre o tamanho dos grupos de primatas observados na
natureza e o tamanho relativo do neocórtex. As técnicas para este estudo são
as da área de Mecânica Estatística, incluindo Campo Médio e Método de Monte Carlo.
As interações são modeladas a partir de teoria de aprendizado de máquinas, com considerações neurocognitivas.
 
Neste projeto queremos estudar as consequências da transição de quebra de simetria na complexificação da sociedade.
Recentemente \cite{Sassaman} houve críticas sérias à emergência de
complexidade das sociedades causada por razões ambientais.  
Resultados de pesquisas paleoambientais dão apoio a hipótese de que mudanças de organização
 econômica foram causadas por mudanças estruturais na organização social
que levaram a desigualdades, e não como descrito na teoria de origem
do dinheiro de Menger onde primeiro há complexificação e depois o aparecimento
de desigualdades. No modelo que estudaremos há primeiro a quebra de simetria
para posterior aparecimento da complexidade economica.
 estudo do  modelo de quebra de simetria da organização social
devido a limites cognitivos está bastante avançado. Neste projeto, buscamos adicionar um ingrediente novo,
que consiste em permitir a troca de diferentes tipos de mercadorias ou favores.

Será estudada a formação de coligações cimentadas por estas trocas, num ambiente competitivo. Novamente limites cognitivos impõem restrições ao
conhecimento total da rede de relações sociais. As desvantagens destas restrições se devem ao fato de que o conhecimento imperfeito do meio social leva a uma
desvantagem ou custo social. A competição leva a estrutura da representação
da rede social numa topologia estrela, que mostra a organização social
hierárquica. A simulação de uma sociedade de troca de mercadorias, com probabilidade
dependente da distância social introduz uma dinâmica que pode ou não levar,
dependendo da escolha de parâmetros, ao aparecimento de uma mercadoria central
de trocabilidade universal.

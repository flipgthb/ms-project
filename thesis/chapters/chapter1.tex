
%%% Capítulo 1 - Introdução

\chapter{Um Panorama do Estudo\\de Fenômenos Sociais}
\label{ch:C1}
\chapquote{
... I can see no other escape from this dilemma (lest our true aim be  lost for ever) than that some of us should venture to embark on a synthesis  of facts and theories, albeit with second-hand and incomplete knowledge of some of them, and at the risk of making fool of ourselves.
So much for my apology.
}{Erwin Schrödinger, \emph{What is Life?} (1944)}

Os conceitos de fenômeno social e de natureza humana se entrelaçam.
A complexidade e diversidade observadas na sociedade despertam o interesse de cientistas em diferentes campos, dedicados a questionar e entender os mecanismos que as produzem.
É possível, por exemplo, que um sociólogo se interesse em entender como diferentes grupos lidam com resolução de conflitos, enquanto um psicólogo esteja preocupado com a abordagem de cada indivíduo envolvido numa situação de conflito, um neurocientista queira analisar a ativação cerebral em cada abordagem descrita pelo psicólogo, um estatístico queira entender as correlações entre estrutura social e comportamento, e a um físico talvez caiba tentar entender a dinâmica que rege esse tipo de fenômeno.

Uma característica, talvez a mais interessante, de sistemas sociais é a exibição simultânea de estruturas organizadas e ricas em diversidade.
Embora as pessoas tendam a reproduzir o comportamento de seus pares, ainda observamos diversidade cultural, formação de estados e dialetos num mesmo idioma por exemplo.
O estudo dessa característica das sociedades humanas tem sido abordada por diferentes estratégias na pesquisa contemporânea.

Nesse contexto, o psicólogo social \emph{J. Haidt} e seus colegas na formulação da \emph{Teoria dos Fundamentos Morais} \footcite{Haidt2007,Graham2011} estabeleceram as bases para relacionar a resposta neural de indivíduos em questões de conflito moral com a participação deste na manutenção da sociedade.
Através de resultados sobre o aprendizado de comportamento moral, da intuitividade dos julgamentos e do mapeamento da moral, a Teoria dos Fundamentos Morais, propõe que o domínio da moralidade humana não é composto apenas por questões relacionadas ao modo como dois indivíduos devam se tratar, mas também por características que dizem respeito ao papel do indivíduo no grupo.
Além disso, a diversidade encontrada no conceito de moral através das culturas pode ser explicada tomando a moral como um conjunto de mecanismos, biológicos ou culturais, que regulam o egoísmo e tornam a vida social possível.
Essa linha de pensamento contribui para a compreensão da complexidade social na escala das interações humanas e será um ponto de partida útil na elaboração dos modelos estudados neste trabalho.

Numa perspectiva mais coletivista, \emph{R. Axelrod} \footcite{Axelrod1997} propôs um modelo qualitativo para a diversidade cultural em termos de agentes interagentes, no qual ele foi capaz de mostrar a possibilidade da coexistência de diferentes culturas numa dada região como uma consequência da homofilia local.
Nesse modelo, a capacidade de interagir com maior probabilidade com agentes mais parecidos é capaz de produzir fronteiras separando regiões de agentes que são compatíveis e incompatíveis.
Esse resultado é interessante por si, mas o modelo deixa aberto generalizações em diversos sentidos como a topologia do grafo onde os agentes interagem e o efeito das particularidades no mecanismo de aprendizado.

O uso de modelos de agentes em grafos tem sido amplamente empregado no estudo de sistemas sociais.
O presente trabalho se insere numa série de outros no esforço de entender melhor o papel das características humanas na constituição das estruturas sociais.
Integram essa série, modelos relacionando o aprendizado moral com as estruturas cognitivas da interação de indivíduos \footcite{Caticha2011,Vicente2014,Cesar2014} e modelos de surgimento de autoridade e desenvolvimento econômico \footcite{Papa2014,Calsaverini2013}.

O ponto em comum nesses trabalhos recentes são algumas bases empíricas que proporcionam a abordagem de comportamentos sociais como consequência das capacidade humana de aprendizado, das respostas neurológicas a dilemas morais e conflitos sociais ou nas escolhas de estratégias, das estruturas de informação provenientes das interações sociais e de uma compreensão um pouco mais ampla dos processos de aprendizado.

A exploração dos fenômenos sociais como consequência das características psico ou neurológicas dos indivíduos distingue esses recentes esforços das linhas mais \emph{`cinemáticas`} de estudo \footcite{Castellano2009}.
Em outras palavras, além da descrição de como um certo fenômeno ocorre são interessantes também as causas desse fenômeno.
Seguiremos, então, com uma breve apresentação das motivações empíricas sobre o comportamento humano e sua possível influência nas construções sociais.

\section{Aprendizado por Reforço}

Uma das características, não exclusivamente, humanas é sua capacidade de aprender através do sucesso ou do fracasso.
Somos capazes de aprender comportamento, criar abstrações e fazer previsões com base em experiência prévia em diversas circunstâncias.
Embora essa afirmação não seja chocante em si, talvez surpreenda o fato de que estamos aprendendo como funcionam nossos mecanismos de aprendizado, a nível neurológico, psicológico e social.

Um desses mecanismos é o de aprendizado por reforço, ou em outras palavras a habilidade de aprender com erros e acertos.
Para isso é necessário que tenhamos aparatos neurais para planejar uma ação, prever o resultado e detectar erros na previsão para poder ajustar uma próxima ação.

Estudos realizados por \parencite{Holroyd2002a} mostram como o mecanismo de as regiões do Córtex Cingulado Anterior \footnote{tradução livre de \emph{Anterior Cingulate Cortex}(ACC)} em conjunto com os sistemas dopaminérgicos do mesencéfalo funcionam como um mecanismo de detecção de erros e recompensas durante o aprendizado de uma tarefa.
Esse mecanismo é estudado através da análise de eletroencefalograma, mais especificamente na ocorrência de Negatividade Relacionada ao Erro \footnote{tradução livre de \emph{Error-Related Negativity} (ERN)}.
A existência de tal mecanismo de aprendizado por reforço é uma das bases que fundamenta o uso da teoria de aprendizado de máquina para modelar o comportamento social.

A tarefa de detectar conflitos exercida pelo córtex cingulado anterior está diretamente relacionada à sensação de dor física.
Em paralelo à capacidade de aprendizado, os estudos de \parencite{Somerville2006,Eisenberger2003} apontam um aumento na atividade do córtex cingulado anterior quando um indivíduo vivencia exclusão social ou é avaliado negativamente por outras pessoas.
O fato da região do cérebro responsável pela dor física também ser ativada durante a experiência de conflitos sociais sugere que há um custo, talvez análogo à dor física, a ser pago ao se envolver em tais conflitos.
Essa relação de analogia entre dor física e dor por exclusão social nos permite sustentar a hipótese do \emph{custo cognitivo} que regulará as interações no nosso modelo.

Outro resultado interessante sobre aprendizado, no estudo desenvolvido por \parencite{Amodio2007}, é a correlação entre a amplitude da Negatividade Relacionada ao Erro e o posicionamento político auto declarado de indivíduos.
Essa diferença na amplitude da ativação se reflete na sensitividade do indivíduo para responder na resposta de conflitos.
Em particular, foi observado que conservadores tendem a apresentar menores amplitudes de ERN quando comparados a liberais.
Dada a relevância da detecção e resolução de conflitos, é de se esperar que liberais e conservadores \footnote{essas denominações de \emph{liberal} e \emph{conservador} são relativas ao espectro político do Estados Unidos, embora resultado citado seja de aplicabilidade mais ampla} apresentem diferentes \emph{estilos cognitivos} de aprendizado, o usaremos essa abordagem ao formular o custo cognitivo.

\section{Evidências do Aprendizado Social}

A formação da opinião de um indivíduo com relação à algum dilema é influenciada não apenas pela informação objetiva coletada \footnotemark pelo seu aparato sensorial, \footnotetext{ou ao menos tão objetiva quanto possível, dadas as limitações biológicas} como  também pela \emph{pressão social} de indivíduos empenhados na discussão de tal dilema.
O conteúdo dessa afirmação foi o tema de estudos de \emph{M. Sherif} e \emph{S. Asch} \footcite{Sherif1937, Asch1955}.

O experimento de \emph{Sherif} é estabelecido da seguinte forma: um grupo de participantes é colocado numa sala totalmente escura exceto por um ponto de luz projetado a uma distância fixa de um grupo de participantes.
Dentre eles, dois estão de prévio acordo, sem o conhecimento do terceiro, em estimar um certo deslocamento para o ponto de luz dentro de um certo intervalo de valores.
Após alguns segundos observando o ponto, que está parado, os participantes revelam suas estimativas para o deslocamento do ponto de luz, fenômeno conhecido como ilusão autocinética, e uma nova rodada de observações é feita.
No dia seguinte, o experimento é repetido apenas com o terceiro participante.
A hipótese é que o número de observações que o participante \emph{`ingênuo`} faz dentro do intervalo predeterminado pelos experimentadores é influência da interação entre os participantes.

Os resultados desse experimento mostram que as estimativas do participante ingênuo convergem para o mesmo valor das do grupo, caindo dentro do intervalo determinado, dependendo do quão grosseira é a percepção do indivíduo sobre essa estimativa do grupo.
Além disso, a estimativa persiste durante o experimento individual.

O experimento realizado por \emph{Asch} tem como objetivo determinar como esse tipo de influência varia conforme a variação do tamanho do grupo, ou da precisão da opinião do grupo com relação ao objetivo entre outras circunstâncias.
Para isso, grupo entre seis e 9 indivíduos, sendo todos com exceção de um confederados do estudo, aos quais eram apresentadas duas cartas.
Uma carta continha uma barra preta e a outra continha três barras pretas de diferentes comprimentos, das quais os participantes deveriam escolher aquela com o mesmo comprimento da barra da outra carta.
Sempre uma das três barra era a correta, e o experimento foi repetido com uma variedade de diferenças relativas entre as três barras, de modo a regular a dificuldade da tarefa.

Para verificar a influência do grupo, que era instruído a escolher uma das opções erradas um certo número de vezes, foi medido o número de vezes que o indivíduo ingênuo errava na escolha da barra correta quando em grupo e de forma individual.
Observou-se que a taxa de erro, tipicamente da ordem de $1\%$ ao realizar a tarefa individualmente, aumentava para cerca de $30\%$ quando realizada sob na presença do grupo.

Um estudo mais recente, \parencite{Campbell-Meiklejohn2010} mostrou através da análise de imagens de ressonância magnética funcional o efeito da influência social na atribuição de valor a algo.
A atribuição de valor está associada à atividade da região do estriado ventral no cérebro, e a influência social gera um estímulo nessa região em indivíduos interagindo durante uma tarefa que demande a avaliação de alguma quantidade.

Esses estudos mostram o efeito direto da influência do grupo na constituição da opinião de indivíduos.
Aliada ao conhecimento sobre os mecanismos de aprendizado por reforço, de dor por exclusão social e aos diferentes estilos cognitivos, a influência social formará o mecanismo de aprendizado de comportamento social.
O ingrediente final para a construção do modelo diz respeito sobre a \emph{`codificação`} do espaço no qual existe um certo comportamento.
Mais especificamente, tomaremos como referência o mapeamento da moral humana e a intuitividade dos julgamentos morais, apresentados a seguir.

\section{Moral e a Primazia da Intuição}

A Teoria do Fundamentos morais desenvolvida por \emph{J. Haidt} e \emph{J. Graham} \footcite{Haidt2007, Graham2011} contribui para a compreensão da moral humana através de quatro princípios, dos quais dois são essenciais na elaboração deste trabalho.

O primeiro diz respeito à primazia da intuição, ou seja, ao modo automático e inconsciente de tomar decisões morais.
O uso da palavra \emph{primazia} tem como principal objetivo lembrar que, embora a intuição seja a primeira fonte de conclusões, ela não necessariamente domina o julgamento do indivíduo, podendo ser corrigida posteriormente por mecanismos a nível da consciência e do raciocínio.

Essa característica da intuitividade nos permite fazer a associação entre a evolução da cognição moral e a resposta a um estímulo adequado.
Essa associação proporciona um dos vínculos interpretativos entre o aprendizado de comportamento social e a teoria do aprendizado de máquina.

O segundo princípio importante para nós é o mapeamento das características morais num conjunto de \emph{fundamentos universais} da moral humana.
Isso significa que, embora os valores morais variem entre culturas e épocas da civilização, há características em comum entre todas elas.
Essas características são, de certa forma, independentes entre si e formam uma base sobre a qual a moral é construída dentro da população.
Em particular, \emph{Haidt} e seus colegas determinaram através da análise de questionários aplicados sobre uma vasta extensão do planeta que são apenas os fundamentos da moral humana, a saber Violência/Cuidado, Justiça/Reciprocidade, Inclusão em Grupo/Lealdade, Autoridade/Respeito e Pureza/Santidade.

A contribuição desse princípio para o nosso modelo é a justificativa na representação de características cognitivas dos indivíduos através de um vetor em algum espaço adequado.
Note que, embora a Teoria dos Fundamentos Morais verse apenas sobre a moralidade humana, faremos uma extrapolação e aplicaremos esse conceito sobre alguma característica cognitiva qualquer, na esperança de prover algum modelo que seja capaz de fazer previsões.

As considerações feitas aqui não ilustram a completeza de cada  referência feita, que trazem resultados interessantes por si.
Entretanto, as motivações empíricas apresentadas são suficientes para o desenvolvimento do trabalho, com exceção de algum resultados específicos da teoria do aprendizado de máquinas, que serão apresentados no capítulo a seguir.
Daremos, portanto, seguimento ao desenvolvimento da teoria, sem nos aprofundarmos mais nesses assuntos.

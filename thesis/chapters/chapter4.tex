%%% Capítulo 4 - Conclusão e Discussão

\chapter{Conclusões e Futuros Estudos}
\label{ch:C4}

\section{Uma Breve Revisão dos Resultados}

Neste trabalho desenvolvemos modelos para o estudo de fenômenos sociais com base em características cognitivas dos humanos.
Como foi visto no capítulo \ref{ch:C1}, o crescente corpo de resultados a respeito do comportamento humano, tanto a nível neural quanto psicossocial nos guiam na captura dos elementos básico do comportamento social.
Estabelecida a base empírica e fazendo uso das técnicas da mecânica estatística e da teoria de aprendizado de máquina, fomos capazes de estabelecer uma dinâmica para o aprendizado social, que exerce papel fundamental na interpretação dos fenômenos exibidos pelos modelos.

No capítulo \ref{ch:C2} desenvolvemos a teoria elementar para a abordagem estatística de fenômenos de interação social através da troca de opiniões.
O uso de ferramentas, como o método de máxima entropia, nos permitiu compreender através da análise de campo médio qual seria o comportamento de agentes que aprendem em conjunto, bem como o de agentes que se antagonizam, em analogia com sistemas ferromagnéticos e antiferromagnéticos.
Note que, embora este trabalho tenha tomado como modelo base outros trabalhos recentes na mesma área \footcite{Cesar2014,Vicente2014,Caticha2011}, esta é a primeira vez que uma análise um pouco mais profunda é feita em relação à desconfiança $\ns$ ao desenvolver a sociedade baseada no custo cognitivo \eqref{eq:Vij}.

No capítulo \ref{ch:C3} conseguimos comparar os resultados de campo médio com simulações usando o custo cognitivo associado à função de modulação Bayesiana, reproduzindo os resultados encontrado nas referências.
Com base nesses resultados, elaboramos um modelo de construção (ou aniquilação) das relações sociais com base na troca de opiniões.
Esse modelo se mostrou capaz de produzir a estrutura de comunidade com distintas opiniões, proporcionando inclusive uma medida do grau de organização dessa estrutura.

Por fim, um pequeno modelo de comportamento do plenário brasileiro foi elaborado para ilustrar algumas das possibilidades ao utilizar a nossa abordagem no estudo de fenômenos reais.
Embora as bases sob as quais esse modelo foi construído demandem mais estudo, seus resultados são interessantes pela capacidade de reproduzir o comportamento observado nas votação em plenário.
Entretanto, esses resultados são tratados como um guia qualitativo e qualquer esperança de usar esse modelo para explicar quantitativamente fenômenos reais necessita de base experimental.

Além dos resultados apresentados, diversos pontos podem ser destacados com relação ao modelo utilizado.
A escolha de usar apenas uma questão é uma limitação que pode ser facilmente removida, embora o custo computacional envolvido é um preço alto a pagar.
A escolha da topologia da rede social como um grafo completo é irreal na maior parte das redes sociais.
Porém, como um dos focos do trabalho era a construção das estruturas sociais, a escolha de alguma topologia mais realista não é facilmente justificável.
Por fim, os resultados apresentados no capítulo \ref{ch:C3} dependem de parâmetros usado nas simulações de Monte Carlo para os quais há pouca margem para interpretação dentro do modelo.

Por exemplo, o raio do cone dentro do qual o vetor cognitivo de um agente pode ser proposto está associado com a velocidade de convergência do algoritmo e pode também ser associado a uma velocidade de aprendizado.
Porém, essa taxa não surge na dedução apresentada no capítulo \ref{ch:C2} e precisaria ser introduzida artificialmente tendo como justificativa seu uso no algoritmo.
Os resultados apresentados neste trabalho são aqueles mais robustos com relação a esses parâmetros, e diversas variações dos modelos que poderiam gerar outras interpretações para os fenômenos não foram apresentados por não estarem sob controle dos parâmetros fornecidos pela teoria.

\section{O que há pela frente?}

Embora o fenômeno de estruturação social possa ser um pouco melhor compreendido através dos resultados deste trabalho, diversas críticas podem ser feitas com relação à elaboração do modelo.
Esforço deve ser desempenhado no sentido de justificar a forma da dinâmica das relações sociais que, diferente da dinâmica de aprendizado, foi estabelecida de forma heurística.
Tal justificativa demanda não apenas uma análise matemática mais delicada das interações sociais como também de resultados experimentais diretamente relacionado a escolha de amigos ou de colegas de trabalho e outros comportamentos relacionados.

A falta do respaldo experimental é outra crítica que pode ser feita a este trabalho.
A busca de resultados em áreas bem estudadas do comportamento social é prioridade para a sequência do trabalho apresentado aqui.
